\documentclass[]{article}
\usepackage{graphicx}
\usepackage{indentfirst}

\usepackage{clrscode3e}
\setlength{\parindent}{0pt}

%clrs code is good!

\title{HW04 for ECE 9343}
\author{Tongda XU, N18100977}

\begin{document}

\maketitle

\section{Question 1: CLRS Exercise 22.1-3}

\begin{codebox}
	\Procname{$\proc{Transpose($Adjlist$)}$}
	\li new $AdjlistPrime$
	\li \For each $node$ in Adjlist
	\li		\Do \For each $subnode$ in $Adjlist(node)$
	\li 		\Do $AdjlistPrime(subnode).insert(node)$
	\li $Adjlist \gets AdjlistPrime$
	\End
\end{codebox}

For adjacent list: just traverse every node and rebuild one\\
$\Theta (E+V)$ for time and space complexity, hard to do it inplace\\

\begin{codebox}
	\Procname{$\proc{Transpose($Adjmatrix$)}$}
	\li \For each $pair(i,j)$ in upper left Adjmatrix
	\li		\Do \proc{Swap($Adjmatrix[i,j], Adjmatrix[j,i]$)}
	\End
\end{codebox}

For adjacent matrix: just transpose the matrix\\
$\Theta (V^2)$ for time and $\Theta(1)$ for space

\section{Question 2: CLRS Exercise 22.1-5}

For adjacent list, it is hard. We should regard it as a \proc{Breadth-first-search($G$)} end at $d = 2$:

\begin{codebox}
	
	\Procname{$\proc{square($G$)}$}
	\li \For each $u$ in $G.vertices$
	\li 	\Do $G.reset()$
	\li     $list = \emptyset$
	\li 	$u.adjlist^{'} = \proc{BFS-Aid($G, u, list, 0$)}$
	\End
	\End
\end{codebox}

\begin{codebox}
	
	\Procname{$\proc{BFS-Aid($G, u, list, dist$)}$}
	\li \For each $v$ in $u.adjlist$
	\li 	\Do \If $v.color = white$ and $dist \le 2$ 
	\li			\Then $list.insert(u)$
	\li 		\proc{BFS-Aid($G, v, list, dist+1$)}
	\End=
	\End
	\li \Return $list$
\end{codebox}

This could cost $\Theta(V^2 + VE)$ time and $\Theta(V+E)$ space (if optimized).\\

For adjacent matrix, the square process would be simple. for each index m of matrix row, if matrix[m][n] exist, calculate bool union of matrix[m] and matrix[n]:

\begin{codebox}
	
	\Procname{$\proc{square($G$)}$}
	\li \For each $m$ in $G.adjMatrix$
	\li \Do \For each $n$ $G.adjMatrix[m]$
	\li 	\Do \If $G.adjMatrix[m][n] == 1$
	\li         \Then $G^{'}.adjMatrix[m]$ = \proc{And($G.adjMatrix[m],G.adjMatrix[n]$)}
	\End
	\End
	\End
	\li \Return $G^{'}$
\end{codebox}

The \proc{square($G$)} cost $\Theta(V^3)$ time and $\Theta(V)$ space (if optimize)

\section{Question 3: CLRS Exercise 22.2-6}

Consider the following condition in Figure ~\ref{fig:22.2-6}: \\
$E_{\pi} = <s, u1>, <u1, v1>, <s, u2>, <u2, v2>$\\
In BFS Tree, $\delta (s, v1), \delta (s, v2) $ is either $<s, u1, v1>, <s, u1, v2>$ or $<s, u2, v1>, <s, u2, v2>$

\begin{figure}
	\centering
	\includegraphics[width=\linewidth]{2226}
	\caption{22.2-6}
	\label{fig:22.2-6}
\end{figure}

\section{Question 4: Traverse Edge of undirected graph}
According to $Theorem 22.10$, all edges are either tree edge or back edge. Modify the \proc{DFS-Visit($G,u$)}, add a \proc{print-path($G,u$)} would do it. Assume a $root = u$ is selected:

\begin{codebox}
	
	\Procname{$\proc{DFS-Visit($G,u$)}$}
	\li $u.color = grey$
	\li $dict[(Vertex,Vertex), edgeType] = \emptyset$
	\li	\For each $v$ in $u.adjList$
	\li \Do \If $v.color == white$
	\li 	\Then $dict(u,v) = treeEdge$
	\li 		  \proc{DFS-Visit($G,v$)}
	\li 	\Else $dict(u,v) = backEgde$
	\End
	\End
	\li \proc{print-path($G,u$)}
	
\end{codebox}

\begin{codebox}
	
	\Procname{$\proc{print-path($G,u$)}$}
	\li \proc{print ($"u"$)}
	\li	\For each $v$ in $u.adjList$
	\li \Do \If $(u,v) == treeedge$
	\li 	\Then \proc{print ($"\rightarrow"$)}
	\li     \proc{print-path($G,v$)}
	\li 	\Else \proc{print ($"\rightarrow v"$)}
	\End
	\End
	
\end{codebox}

line 4,6 cost same level of time as the comparison in line 3, would not change the $\Theta(V+E)$ time complexity of \proc{DFS($G$)}

the print path function as:

This procedure cost $\Theta(V+E)$ as well

\section{Question 5: CLRS Exercise 22.3-12}

Tweak the \proc{DFS-Visit($G,u$)} and \proc{DFS($G$)} would be enough:

\begin{codebox}
	
	\Procname{$\proc{DFS($G$)}$}
	\li	\For each $u$ in $G.V$
	\li \Do $u.color = white$
	\End
	\li $c=1$
	\li	\For each $u$ in $G.V$
	\li \Do \If $u.color = white$
	\li \Then \proc{DFS-Visit($G,u,c$)}
	\li $c$++
	\End
	\End
	
\end{codebox}

\begin{codebox}
	
	\Procname{$\proc{DFS-Visit($G,u,c$)}$}
	\li $u.color = grey$
	\li $u.cc = c$
	\li	\For each $v$ in $u.adjList$
	\li \Do \If $v.color == white$
	\li 	\Then \proc{DFS-Visit($G,v$)}
	\End
	\End
	
\end{codebox}

\proc{DFS($G$)} could be tweaked to do it as well

\section{Question 6: CLRS Exercise 22.4-1}

$ p[27:28] \rightarrow n[21:26] \rightarrow o[22:25] \rightarrow s[23:24] \rightarrow\\
m[1:20] \rightarrow r[6:19] \rightarrow y[9:18] \rightarrow v[10:17] \rightarrow x[15:16] \rightarrow \\
w[11:14] \rightarrow z[12:13] \rightarrow u[7:8] \rightarrow q[2:5] \rightarrow t[3:4]$

\section{Question 7: Show process of SCC}

See Figure~\ref{fig:hw04Ex7}

\begin{figure}
	\centering
	\includegraphics[width=\linewidth]{hw04Ex7}
	\caption{22.2-6}
	\label{fig:hw04Ex7}
\end{figure}

\section{Question 8: CLRS Problem Set 22.1}

\subsection{a-1}

Suppose $(v,u)$ is a backedge. u is ancestor elder than parent of v. This means $(s,u)$ + forwardEdge is shorter than $(s,v)$ produced by BFS which is $\delta(s,v)$ by \textbf{Theorem 22.5}. Same reason for forward edge.   

\subsection{a-2}

By \textbf{Theorem 22.5} $\delta(s,v) = \delta(s,v.parent) + (v.parent, v) = \delta(s,u) + (u,v) \rightarrow v.d = u.d + 1$

\subsection{a-3}

$v.d \le u.d + 1$ : Same as a-1, if $v.d > u.d + 1$, $\delta(s,v) = (s,u) + cross$ instead of $(s,v)$. \\
$v.d \ge u.d$: If $v.d < u.d$, $(v,u)$ should be find out first, since this is undirected graph. 

\subsection{b-1}

Same as a-1, the $(s, u) + backEdge$ would be shorter than $(s, v)$

\subsection{b-2}

Same as a-2, By \textbf{Theorem 22.5} $\delta(s,v) = \delta(s,v.parent) + (v.parent, v) = \delta(s,u) + (u,v) \rightarrow v.d = u.d + 1$

\subsection{b-3}

Only the first half of a-3. if $v.d > u.d + 1$, $\delta(s,v) = (s,u) + cross$ instead of $(s,v)$. 
\subsection{b-4}

By \textbf{Corollary 22.4} and By \textbf{Theorem 22.5}, we know that if v is an ancestor of u $\delta (s, u) = \delta(s, v) + k \rightarrow \delta(s, u) > \delta (s, v) \rightarrow u.d > v.d$, I did not see how $u.d = v.d$ but the statement is correct.

\section{Question 8: CLRS Problem Set 22.3}


\subsection{1. proof}

Euler tour exist $\rightarrow$ in-degree == out-degree: Suppose the cycle through i vertex n times would be $E-cycle = \{ v_{i}, v_{j}, v_{k}, ... , v_{i} \}$. The in-degree of $v_{j}$ would be the time of  $v_{j}$ appears with element in front, and out-degree of $v_{j}$ would be the time of $v_{j}$ appears with element in the back. If $v_{j}$ is not head or tail, this is obvious that every time $v_{j}$ appear, there is element in front and tail. If $v_{j}$ is head, it must also be tail, which balance the in-degree and out-degree again.\\

in-degree == out-degree $\rightarrow$ Euler tour exist

\subsection{2. implement}

This is very similar to SCC, we find closed cycle first then join them with other edge set. This procedure would return a cycle, which is a list of vertex. closed cycle has $cycle.begin() = cycle.end()$, open cycle(path, not a cycle) do not has it. But if Euler tour exist, open cycle would join close cycle into a big cycle.

\begin{codebox}
	
	\Procname{$\proc{CircleFind($cycle,u,v$)}$}
	\li $ClosedCycleSet, OpenCycleSet = \emptyset$
	\li \While $all adjList != \emptyset$ \Do
	\li \For $v$ in $Vertex$ with $adjList != \emptyset$\Do
	\li \If $v.adjList != \emptyset$ \Then
	\li new $cycle = \emptyset$
	\li \proc{CircleFindAid($cycle,u,NIL$)}
	\li \If $cycle.type == closed$
	\li \Then $ClosedCycleSet.push(cycle)$
	\li \Else $OpenCycleSet.push(cycle)$
	
\end{codebox}

\begin{codebox}
	
	\Procname{$\proc{CircleFindAid($cycle,u,v$)}$}
	\li $cycle.insert(u)$
	\li \If $v != NIL$ 
	\li \Then $v.adjList.erase(u)$ \End
	\li \If $u == NIL$ 
	\li \Then $cycle.type = open$
	\li \Return  
	\li \ElseIf $u == cycle.start$ 
	\li \Then $cycle.type = close$
	\li \Return 
	\li \Else $v = u$
	\li $u = u.adjList.begin()$
	\li \proc{CircleFind(cycle, u, v)}
	\End
	\End
	
\end{codebox}

It is easy to find that as we remove an edge from adjacent list once we find it, and we traverse every edge, the time complexity would be $\Theta(E)$

\end{document}
