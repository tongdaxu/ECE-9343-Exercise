\documentclass[]{article}
\usepackage{graphicx}
\usepackage{indentfirst}

\usepackage{clrscode3e}

%clrs code is good!

\title{HW02 for ECE 9343}
\author{Tongda XU, N18100977}

\begin{document}

\maketitle

\section{Question 1: 3-divide maximum subarray}

\begin{codebox}
	\Procname{$\proc{maxFromLeft($A, p, r$)}$}
	\li $max = -\infty$
	\li \For $i \gets p$ \To $r$
	\li		\Do $max = Sum(A,p,i)>max?Sum(A,p,i):max$
 		\End
	\li \Return $max$
\end{codebox}

\begin{codebox}
	\Procname{$\proc{maxFromRight($A, p, r$)}$}
	\li $max = -\infty$
	\li \For $i \gets r$ \Downto $p$
	\li		\Do $max = Sum(A,i,r)>max?Sum(A,i,r):max$
	\End
	\li \Return $max$
\end{codebox}

\begin{codebox}
	\Procname{$\proc{THREE-FOLD-MAXSUB($A, p, r$)}$}
	\li $s = \left \lfloor (p+r)/3 \right \rfloor$
	\li $t = \left \lfloor (p+r)2/3 \right \rfloor$
	\li \If $Sum(A,s,t-1) > 0$
	\li 	\Then \Return $max(maxFromLeft(A, p, s-1), maxFromRight(A, t, r)) + Sum(A,s,t-1)$
	\li	\Else \Return $max(maxFromLeft(A, p, s-1), maxFromRight(A, t, r))$
	\End
\end{codebox}

The time complexity is $\Theta(n)$

\section{Question 2: Intermediate Sequence}

\begin{codebox}
	\Procname{$\proc{bubble sort($A$)}$}
	\li $A = [11, 8, 7, 5, 3, 1] $
	\li $\rightarrow [8, 11, 7, 5, 3, 1] \rightarrow [8, 7, 11, 5, 3, 1] \rightarrow [8, 7, 5, 11, 3, 1] 
	\rightarrow [8, 7, 5, 3, 11, 1] \rightarrow [8, 7, 5, 3, 1, 11]$
	\li $\rightarrow [7, 8, 5, 3, 1, 11] \rightarrow [7, 5, 8, 3, 1, 11] \rightarrow [7, 5, 3, 8, 1, 11]\rightarrow [7, 5, 3, 1, 8, 11]$
	\li $\rightarrow [5, 7, 3, 1, 8, 11] \rightarrow [5, 3, 7, 1, 8, 11] \rightarrow [5, 3, 1, 7, 8, 11]$	
	\li $\rightarrow [3, 5, 1, 7, 8, 11]\rightarrow [3, 1, 5, 7, 8, 11] $
	\li $\rightarrow [1, 3, 5, 7, 8, 11]$
\end{codebox}

\begin{codebox}
	\Procname{$\proc{insertion sort($A$)}$}
	\li $A = [11, 8, 7, 5, 3, 1] $
	\li $\rightarrow [8, 11, 7, 5, 3, 1]$
	\li $\rightarrow [8, 7, 11, 5, 3, 1] \rightarrow [7, 8, 11, 5, 3, 1]$
	\li $\rightarrow [7, 8, 5, 11, 3, 1] \rightarrow [7, 5, 8, 11, 3, 1] \rightarrow [5, 7, 8, 11, 3, 1]$
	\li $\rightarrow [5, 7, 8, 3, 11, 1] \rightarrow [5, 7, 3, 8, 11, 1] \rightarrow [5, 3, 7, 8, 11, 1] \rightarrow [3, 5, 7, 8, 11, 1]$
	\li $\rightarrow [3, 5, 7, 8, 1, 11] \rightarrow [3, 5, 7, 1, 8, 11] \rightarrow [3, 5, 1, 7, 8, 11] 
	\rightarrow [3, 1, 5, 7, 8, 11] \rightarrow [1, 3, 5, 7, 8, 11]$
\end{codebox}

\section{Question 3: Illustrate Merge Sort}

\begin{codebox}
	\Procname{$\proc{Merge sort($A$)}$}
	\li $15, 16, 25, 29, 30, 40, 48$
	\li $15, 29, 48 || 16, 25, 30, 40$
	\li $29 || 15, 48 || 25, 40 || 16, 30$
	\li $- || 48 || 15 || 40 || 25 || 16 || 30$
\end{codebox}


\section{Question 4: CLRS Problem 2-1}
\subsection{a. show time complexity}
$\Theta(T) = \frac{n}{k}\Theta(n^{2}) = \Theta(nk)$
\subsection{b. show merge, c. show whole}

There should not be anything special about Merge function, just use the original interface and implement of Merge in CLRS pp 31.\\
$$ T(n)=\left\{
\begin{array}{lcl}
 n       &      & {n \le k}\\
2T(\frac{1}{2}n) + n     &      & {n > k}\\
\end{array} \right. $$

Regarding the iterative tree, it is easy to notice that:
For branch (Merge), the complexity is $\Theta(nlg\frac{n}{k})$, For leaf (Insertion sort), is $\Theta(nk)$
\begin{codebox}
	\Procname{$\proc{Merge-sort($A, p, r, k$)}$}
	\li \If $r-p + 1 \le k$
	\li \Then   Insertion-Sort$(A,p,r)$
	\li 		\Return
	\li \ElseIf $p < r$
	\li \Then	$q = \left \lfloor (p+r)/2 \right \rfloor$
	\li 		Merge-Sort($A, p, q$)
	\li			Merge-Sort($A, q+1, r$)
	\li			Merge ($A, p, q, r$)
	\li			\Return
	\li \Else  \Return
		\End
\end{codebox}

\section{Question 5: verify}
Proof: $ T(n) = O(n)$ \\
Suppose $\forall k<n, \exists c_{2}, T(k) \le c_{2}k - 10\\
\rightarrow T(n) = c_{2}\alpha n + c_{2}(1-\alpha) n - 20 + 10 \le c_{2}n - 10\\
\rightarrow T(n) = O(c_{2}n-10)\\
\rightarrow T(n) = O(n)\\$
\\
Proof: $ T(n) = \Omega(n)$ \\
Suppose $\forall k<n, \exists c_{1}, T(k) \ge c_{1}k\\
\rightarrow T(n) = c_{1}\aleph n + c_{1}(1-\alpha) n + 10 \ge c_{1}n\\
\rightarrow T(n) = \Omega(n)\\$
\\
$T(n) = O(n), T(n) = \Omega(n) \rightarrow T(n) = \Theta(n)$

\section{Question 6: solve and verify}
Notice that $TreeHeight=h = log_{\frac{3}{2}}n$\\
For branch $\Theta(n) = \sum _{1}^{h + 1} n(\frac{4}{3})^h = n\frac{(\frac{4}{3})^h - 1}{\frac{4}{3} - 1} = \Theta(n^{\frac{ln2}{ln3-ln2}})$\\
For leaf $\Theta(n) = 2^h = \Theta(n^{\frac{ln2}{ln3-ln2}})$\\
$\rightarrow T(n) = \Theta(n^{\frac{ln2}{ln3-ln2}}) = \Theta(n^{log_{\frac{3}{2}}2}) = \Theta(2^{log_{\frac{3}{2}}n})$\\
\\
Proof: $ T(n) = O(2^{log_{\frac{3}{2}}n}) - 3n$ \\
Suppose $\forall k<n, \exists c_{2}, T(k) \le c_{2}2^{log_{\frac{3}{2}}n} - 3n\\
\rightarrow T(n) = c_{2}2*2^{log_{\frac{3}{2}}\frac{2}{3}n} -4n + n \le c_{2}2^{log_{\frac{3}{2}}n} -3n\\
\rightarrow T(n) =  O(2^{log_{\frac{3}{2}}n} -3n)\\
\rightarrow T(n) =  O(2^{log_{\frac{3}{2}}n})\\$
\\
Proof: $ T(n) = \Omega(n^{log_{\frac{3}{2}}2})$ \\
Suppose $\forall k<n, \exists c_{1}, T(k) \ge c_{1}n^{log_{\frac{3}{2}}2}\\
\rightarrow T(n) = c_{1}2(\frac{2}{3}n)^{log_{\frac{3}{2}}2} + \frac{4}{3}n
 = c_{1}2*(\frac{3}{2})^{log_{\frac{3}{2}}2}*n^{log_{\frac{3}{2}}2} + \frac{4}{3}n = c_{1}n^{log_{\frac{3}{2}}2} + \frac{4}{3}n \ge c_{1}n^{log_{\frac{3}{2}}2} + n\\
\rightarrow T(n) = \Omega(n^{log_{\frac{3}{2}}2})\\$
\\
$T(n) =  O(2^{log_{\frac{3}{2}}n}), T(n) = \Omega(n^{log_{\frac{3}{2}}2}) \rightarrow T(n) = \Theta(n^{log_{\frac{3}{2}}2})$
\section{Question 7: solve and verify}

Notice that for iterative tree: \\
$\Theta (n^2) = 2T(\frac{1}{4}n) + n^2 \le T(n) \le 2T(\frac{1}{2}n) + n^2 = \Theta (n^2)\\$
\\
Proof: $T(n) = O(n^2)$, Suppose $\forall k < n, T(k) = O(k^2)\\
\rightarrow \exists c_{2}>\frac{16}{11}, T(k) \le c_{2}n^2, T(n) \le (\frac{5}{16}c_{2} + 1)n^{2} \le c_{2}n^2, c_{2}>\frac{16}{11} \\$
\\
Proof: $T(n) = \Omega(n^2)$, Suppose $\forall k < n, T(k) = \Omega(k^2)\\
\rightarrow \exists c_{1}<\frac{16}{11}, T(k) \ge c_{1}n^2, T(n) \ge (\frac{5}{16}c_{1} + 1)n^{2} \ge c_{1}n^2, c_{1}<\frac{16}{11} \\
\rightarrow T(n) = \Theta(n^2)$\\

\section{Question 8: solve}
Let $n = 2^m$, Then $T(2^m) = 9T(2^{\frac{m}{6}}) + m^2\\
\rightarrow S(m) = 9S(\frac{1}{6}m) + m^2\\$
From Branch: $S(m) = m^{log_{6}\frac{3}{2} + 2}$\\
From Leave: $S(m) = m^{log_{6}9}$\\
So, $S(m) = \Theta (m^{log_{6}\frac{3}{2} + 2}) \\
\rightarrow T(n) = T(2^m) = \Theta((lg(n))^{log_{6}\frac{3}{2} + 2})$

\section{Question 9: solve and justify}
\subsection{a}
For leaf $\Theta(n) = n^{log_{3}2}$\\
For branch, Notice that $n^{\frac{1}{2}} <n^{\frac{1}{2}}lgn < n^{\frac{1}{2} + \epsilon}\\
\rightarrow 2S(\frac{1}{3}n) + n^{\frac{1}{2}} < T(n) < S(n) = 2S(\frac{1}{3}n) + n^{\frac{1}{2} + \epsilon}$\\
Notice that the branch complexity of $\Theta(n^{log_{3}2}) \le S(n) = n^{\frac{1}{2}+\epsilon} \le \Theta(n^{log_{3}2})$\\
$\rightarrow T(n)$ is equally dominated by branch and leaf, $T(n) = \Theta(n^{log_{3}2})$


\subsection{b}
For branch $T(n) = \Theta(hn^2) = \Theta(lognn^2)$\\
For leaf $T(n) = \Theta(n^2)$\\
$\rightarrow T(n)$ is dominated by branch, $T(n) =\Theta(lognn^2)$

\subsection{c}
For leaf $T(n) = \Theta(4^{log_{2}n}) = \Theta(n^2)$\\
For branch, notice that $4*(\frac{1}{2})^{\frac{5}{2}} = 2^{-\frac{1}{2}} < 1, T(n) = \Theta(n^{\frac{5}{2}})$\\
$\rightarrow T(n)$ is dominated by branch, $T(n) =\Theta(n^{\frac{5}{2}})$

\subsection{d}
For branch $TreeHeight = h = \frac{n}{2}, T(n) = \frac{1}{2}\sum_{1}^{h + 1} \frac{1}{n} = \frac{1}{2}(lnn-ln2) = \Theta(lnn)$\\
For leaf $T(n) = \Theta(c)$\\
$\rightarrow T(n)$ is dominated by branch, $T(n) =\Theta(lnn)$


\end{document}
